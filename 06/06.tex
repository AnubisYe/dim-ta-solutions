
%
% Ragnar Ardal
%
% Logic in Computer Science
%
% Reykjavik University 2016
%

\documentclass[11pt,a4paper]{article}

%\usepackage{multicol}

\usepackage{amsmath}
\usepackage{amssymb}
%\usepackage{amsthm}

\usepackage{mathtools}

\usepackage{enumerate}

% Organization of LaTeX file

\newcommand{\chapter}[2]{%
\setcounter{section}{#1}%
\addtocounter{section}{-1}%
\section{#2}%
}
\newcommand{\subchapter}[2]{%
\setcounter{subsection}{#1}%
\addtocounter{subsection}{-1}%
\subsection{#2}%
}
\newcommand{\problem}[1]{%
\setcounter{subsubsection}{#1}%
\addtocounter{subsubsection}{-1}%
\subsubsection{\hfill}%
}
\newenvironment{subproblem}
	{\begin{enumerate}[a)]}
	{\end{enumerate}}
\newcommand{\skipitem}{\addtocounter{enumi}{1}}
\newcommand{\solution}{%
\subsubsection*{Solution}%
}

% Misc math operators

\DeclarePairedDelimiter{\ceil}{\lceil}{\rceil}
\DeclarePairedDelimiter{\floor}{\lfloor}{\rfloor}

% Operators for predicate logic

\DeclareMathOperator{\T}{\text{\textbf T}}
\DeclareMathOperator{\F}{\text{\textbf F}}
\DeclareMathOperator{\lthen}{\to}
\DeclareMathOperator{\limplies}{\to}
\DeclareMathOperator{\lwhen}{\gets}
\DeclareMathOperator{\lif}{\gets}
\DeclareMathOperator{\liff}{\leftrightarrow}
\DeclareMathOperator{\lxor}{\oplus}

% Sets

\DeclarePairedDelimiter{\set}
	{\lbrace}
	{\rbrace}
\DeclareMathOperator{\ZZ}{\mathbb{Z}}
\DeclareMathOperator{\SetOfIntegers}{\ZZ}
\DeclareMathOperator{\ZZPos}{\mathbb{Z}^+}
\DeclareMathOperator{\SetOfPositiveIntegers}{\ZZPos}
\DeclareMathOperator{\NN}{\mathbb{N}}
\DeclareMathOperator{\SetOfNaturalNumbers}{\NN}
\DeclareMathOperator{\RR}{\mathbb{R}}
\DeclareMathOperator{\SetOfRealNumbers}{\RR}
\DeclareMathOperator{\RRPos}{\mathbb{R}^+}
\DeclareMathOperator{\SetOfPositiveRealNumbers}{\RRPos}
\DeclareMathOperator{\QQ}{\mathbb{Q}}
\DeclareMathOperator{\SetOfRationalNumbers}{\QQ}
\DeclareMathOperator{\CC}{\mathbb{C}}
\DeclareMathOperator{\SetOfComplexNumbers}{\CC}

\begin{document}

\title{TA Solutions 5}
\date{}
\author{}
\maketitle

\chapter{2}{Basic Structures: Sets, Functions, Sequences, Sums, and Matrices}
	\subchapter{6}{Matrices}
		
%
% Worked by AUTHOR
%
% Page numbers:
%  -    p - 7th Custom Edition
%  -    p - Icelandic Edition
%  -  185 - 7th Edition (US)
%  -    p - 6th Edition
%  -    p - 7th Global Edition
%

\problem{27}
	Let
	$$
	\mathbf{A} =
	\begin{bmatrix}
		1 & 0 & 1 \\ 1 & 1 & 0 \\ 0 & 0 & 1
	\end{bmatrix}
	\text{\;and\;}
	\mathbf{B} =
	\begin{bmatrix}
		0 & 1 & 1 \\ 1 & 0 & 1 \\ 1 & 0 & 1
	\end{bmatrix}.
	$$
	Find
	\begin{subproblem}
		\item $\mathbf{A} \lor \mathbf{B}$.
		\item $\mathbf{A} \land \mathbf{B}$.
		\item $\mathbf{A} \odot \mathbf{B}$.
	\end{subproblem}

\solution
	\begin{subproblem}
		\item 
		\item 
		\item 
	\end{subproblem}

		
%
% Worked by AUTHOR
%
% Page numbers:
%  -    p - 7th Custom Edition
%  -    p - Icelandic Edition
%  -  185 - 7th Edition (US)
%  -    p - 6th Edition
%  -    p - 7th Global Edition
%

\problem{29}
	Find the Boolean product of $\mathbf{A}$ and $\mathbf{B}$, where
	$$
	\mathbf{A} =
	\begin{bmatrix}
		1 & 0 & 0 & 1 \\
		0 & 1 & 0 & 1 \\
		1 & 1 & 1 & 1
	\end{bmatrix}
	\text{\;and\;}
	\mathbf{B} =
	\begin{bmatrix}
		1 & 0 \\
		0 & 1 \\
		1 & 1 \\
		1 & 0 \\
	\end{bmatrix}
	$$

\solution


\chapter{4}{Number Theory and Cryptography}
	\subchapter{1}{Divisibility and Modular Arithmetic}
		
%
% Worked by AUTHOR
%
% Page numbers:
%  -    p - 7th Custom Edition
%  -    p - Icelandic Edition
%  -  244 - 7th Edition (US)
%  -    p - 6th Edition
%  -    p - 7th Global Edition
%

\problem{9}
	What are the quotient and remainder when
	\begin{subproblem}
		\item 19 is divided by 7?
		\item $-111$ is divided by 11?
		\skipitem
		\skipitem
		\item 0 is divided by 19?
	\end{subproblem}

\solution

		
%
% Worked by AUTHOR
%
% Page numbers:
%  -    p - 7th Custom Edition
%  -    p - Icelandic Edition
%  -  244 - 7th Edition (US)
%  -    p - 6th Edition
%  -    p - 7th Global Edition
%

\problem{11}
	What time does a 12-hour clock read
	\begin{subproblem}
		\item 80 hours after it reads 11:00?
		\item 40 hours before it reads 12:00?
		\item 100 hours after it reads 6:00?
	\end{subproblem}

\solution
	\begin{subproblem}
		\item 
		\item 
		\item 
	\end{subproblem}


		
%
% Worked by AUTHOR
%
% Page numbers:
%  -    p - 7th Custom Edition
%  -    p - Icelandic Edition
%  -  244 - 7th Edition (US)
%  -    p - 6th Edition
%  -    p - 7th Global Edition
%

\problem{21}
	Evaluate these quantities.
	\begin{subproblem}
		\item $13 \bmod 3$
		\item $-97 \bmod 11$
		\item $155 \bmod 19$
		\item $-221 \bmod 23$
	\end{subproblem}

\solution
	\begin{subproblem}
		\item 
		\item 
		\item 
		\item 
	\end{subproblem}

	\subchapter{3}{Primes and Greatest Common Divisors}
		
%
% Worked by AUTHOR
%
% Page numbers:
%  -    p - 7th Custom Edition
%  -    p - Icelandic Edition
%  -  272 - 7th Edition (US)
%  -    p - 6th Edition
%  -    p - 7th Global Edition
%

\problem{1}
	Determine whether each of these integers is prime.
	\begin{subproblem}
		\item 21
		\skipitem 
		\skipitem 
		\item 97
		\skipitem 
		\item 143
	\end{subproblem}

\solution
	\begin{subproblem}
		\item 
		\skipitem 
		\skipitem 
		\item 
		\skipitem 
		\item 
	\end{subproblem}

		
%
% Worked by AUTHOR
%
% Page numbers:
%  -    p - 7th Custom Edition
%  -    p - Icelandic Edition
%  -  272 - 7th Edition (US)
%  -    p - 6th Edition
%  -    p - 7th Global Edition
%

\problem{3}
	Find the prime factorization of each of these integers.
	\begin{subproblem}
		\item 88
		\skipitem
		\skipitem
		\item 1001
	\end{subproblem}

\solution
	\begin{subproblem}
		\item 
		\skipitem
		\skipitem
		\item
	\end{subproblem}

		
%
% Worked by AUTHOR
%
% Page numbers:
%  -    p - 7th Custom Edition
%  -    p - Icelandic Edition
%  -  272 - 7th Edition (US)
%  -    p - 6th Edition
%  -    p - 7th Global Edition
%

\problem{17}
	Determine wheter the integers in each of these sets are pairwise relatively prime.
	\begin{subproblem}
		\item 11, 15, 19
		\item 14, 15, 21
	\end{subproblem}

\solution
	\begin{subproblem}
		\item 
		\item 
	\end{subproblem}

		
%
% Worked by AUTHOR
%
% Page numbers:
%  -    p - 7th Custom Edition
%  -    p - Icelandic Edition
%  -  273 - 7th Edition (US)
%  -    p - 6th Edition
%  -    p - 7th Global Edition
%

\problem{25}
	What are the greatest common divisors of these pairs of integers?
	\begin{subproblem}
		\item $3^7 \cdot 5^3 \cdot 7^3$, $2^{11} \cdot 3^5 \cdot 5^9$
		\item $11 \cdot 13 \cdot 17$, $2^9 \cdot 3^7 \cdot 5^5 \cdot 7^3$
		\skipitem
		\item $41 \cdot 43 \cdot 53$, $41 \cdot 43 \cdot 53$
	\end{subproblem}

\solution
	\begin{subproblem}
		\item 
		\item 
		\skipitem
		\item 
	\end{subproblem}

		
%
% Worked by AUTHOR
%
% Page numbers:
%  -    p - 7th Custom Edition
%  -    p - Icelandic Edition
%  -  273 - 7th Edition (US)
%  -    p - 6th Edition
%  -    p - 7th Global Edition
%

\problem{33}
	Use the Euclidean algorithm to find
	\begin{subproblem}
		\skipitem
		\item $\gcd(111, 201)$.
		\item $\gcd(1001, 1331)$.
	\end{subproblem}

\solution
	\begin{subproblem}
		\skipitem
		\item 
		\item 
	\end{subproblem}

	\subchapter{5}{Applications of Congruences}
		
%
% Worked by AUTHOR
%
% Page numbers:
%  -    p - 7th Custom Edition
%  -    p - Icelandic Edition
%  -  284 - 7th Edition (US)
%  -    p - 6th Edition
%  -    p - 7th Global Edition
%

\problem{284}
	By inspection (as discussed prior to Example 1), find an inverse of 4 modulo 9.

\solution

		
%
% Worked by AUTHOR
%
% Page numbers:
%  -    p - 7th Custom Edition
%  -    p - Icelandic Edition
%  -  284 - 7th Edition (US)
%  -    p - 6th Edition
%  -    p - 7th Global Edition
%

\problem{7}
	Show that if $a$ and $m$ are relatively prime positive integers,
	then the inverse of $a$ modulo $m$ is unique modulo $m$.
	[Hint: Assume that there are two solutions $b$ and $c$ of the congruence $ax \equiv 1 \pmod{m}$.
	Use Theorem 7 of Section 4.3 to show that $b \equiv c \pmod{m}$.]

\solution


\end{document}

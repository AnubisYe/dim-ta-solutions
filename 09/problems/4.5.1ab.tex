\documentclass[../main.tex]{subfiles}

\begin{document}

\problem{1}
Consider the letter $A$ described in Figure 4.41.
Find the data matrix for the letter obtained by:
\begin{enumerate}[a)]
	\item Rotating the letter through \(\frac{\pi}{4}\) about the origin.
	\item Rotating the letter through \(\frac{\pi}{4}\) about the point \(\begin{bmatrix}1\\2\end{bmatrix}\).
\end{enumerate}

\solution
\begin{enumerate}[a)]
	\item The data matrix for the letter is
		\[
			D
			=
			\begin{bmatrix}
				0 & 6 & 5 & 1 & 3 \\
				0 & 0 & 3 & 3 & 9 \\
			\end{bmatrix}
		\]
		so we rotate it as we have done before
		\begin{align*}
			R_{\pi/4}D
			&=
			\begin{bmatrix}
				\cos\frac{\pi}{4} & -\sin\frac{\pi}{4} \\
				\sin\frac{\pi}{4} & \cos\frac{\pi}{4} \\
			\end{bmatrix}
			D
			\\&=
			\begin{bmatrix}
				\frac{1}{\sqrt{2}} & -\frac{1}{\sqrt{2}} \\
				\frac{1}{\sqrt{2}} & \frac{1}{\sqrt{2}} \\
			\end{bmatrix}
			\begin{bmatrix}
				0 & 6 & 5 & 1 & 3 \\
				0 & 0 & 3 & 3 & 9 \\
			\end{bmatrix}
			\\&=
			\frac{1}{\sqrt{2}}
			\begin{bmatrix}
				1 & -1 \\
				1 & 1 \\
			\end{bmatrix}
			\begin{bmatrix}
				0 & 6 & 5 & 1 & 3 \\
				0 & 0 & 3 & 3 & 9 \\
			\end{bmatrix}
			\\&=
			\frac{1}{\sqrt{2}}
			\begin{bmatrix}
				1\cdot0 + (-1)\cdot0 & 1\cdot0 + 1\cdot0 \\
				1\cdot6 + (-1)\cdot0 & 1\cdot6 + 1\cdot0 \\
				1\cdot5 + (-1)\cdot3 & 1\cdot5 + 1\cdot3 \\
				1\cdot1 + (-1)\cdot3 & 1\cdot1 + 1\cdot3 \\
				1\cdot3 + (-1)\cdot9 & 1\cdot3 + 1\cdot9 \\
			\end{bmatrix}^T
			\\&=
			\frac{1}{\sqrt{2}}
			\begin{bmatrix}
				0 + 0 & 0 + 0 \\
				6 + 0 & 6 + 0 \\
				5 - 3 & 5 + 3 \\
				1 - 3 & 1 + 3 \\
				3 - 9 & 3 + 9 \\
			\end{bmatrix}^T
			\\&=
			\frac{1}{\sqrt{2}}
			\begin{bmatrix}
				0 & 6 & -3 & -2 & -6 \\
				0 & 6 & 8 & 4 & 12 \\
			\end{bmatrix}
			\\&=
			\frac{1}{\sqrt{2}}
			\begin{bmatrix}
				0 & 6 & -3 & -2 & -6 \\
				0 & 6 & 8 & 4 & 12 \\
			\end{bmatrix}
		\end{align*}
	\item We use homogeneous coordinates for the data matrix of the letter
		\[
			K_D
			=
			\begin{bmatrix}
				0 & 6 & 5 & 1 & 3 \\
				0 & 0 & 3 & 3 & 9 \\
				1 & 1 & 1 & 1 & 1 \\
			\end{bmatrix}
		\]
		and since the coordinate we want to rotate around is \(\begin{bmatrix}1\\2\end{bmatrix}\) we get
		\begin{align*}
			&
			\begin{bmatrix*}[r]
				1 & 0 & 1 \\
				0 & 1 & 2 \\
				0 & 0 & 1 \\
			\end{bmatrix*}
			R_{\pi/4}
			\begin{bmatrix*}[r]
				1 & 0 & -1 \\
				0 & 1 & -2 \\
				0 & 0 & 1 \\
			\end{bmatrix*}
			K_D
			\\=&
			\begin{bmatrix*}[r]
				1 & 0 & 1 \\
				0 & 1 & 2 \\
				0 & 0 & 1 \\
			\end{bmatrix*}
			\begin{bmatrix*}[r]
				\frac{1}{\sqrt{2}} & -\frac{1}{\sqrt{2}} & 0 \\
				\frac{1}{\sqrt{2}} & \frac{1}{\sqrt{2}} & 0 \\
				0 & 0 & 1 \\
			\end{bmatrix*}
			\begin{bmatrix*}[r]
				1 & 0 & -1 \\
				0 & 1 & -2 \\
				0 & 0 & 1 \\
			\end{bmatrix*}
			K_D
			\\=&
			\frac{1}{\sqrt{2}}
			\begin{bmatrix*}[r]
				1 & 0 & 1 \\
				0 & 1 & 2 \\
				0 & 0 & 1 \\
			\end{bmatrix*}
			\begin{bmatrix*}[r]
				1 & -1 & 0 \\
				1 & 1 & 0 \\
				0 & 0 & \sqrt{2} \\
			\end{bmatrix*}
			\begin{bmatrix*}[r]
				1 & 0 & -1 \\
				0 & 1 & -2 \\
				0 & 0 & 1 \\
			\end{bmatrix*}
			K_D
			\\=&
			\frac{1}{\sqrt{2}}
			\begin{bmatrix*}[r]
				1 & 0 & 1 \\
				0 & 1 & 2 \\
				0 & 0 & 1 \\
			\end{bmatrix*}
			\begin{bmatrix*}[r]
				1 & -1 & 1 \\
				1 & 1 & -3 \\
				0 & 0 & \sqrt{2} \\
			\end{bmatrix*}
			K_D
			\\=&
			\frac{1}{\sqrt{2}}
			\begin{bmatrix*}[r]
				1 & -1 & 1 + \sqrt{2} \\
				1 & 1 & 2\sqrt{2} - 3 \\
				0 & 0 & \sqrt{2} \\
			\end{bmatrix*}
			K_D
		\end{align*}
\end{enumerate}

\end{document}

\documentclass[11pt,a4paper]{article}

\usepackage{enumerate}
\usepackage{subfiles}

\usepackage{mathtools}
\usepackage{amssymb}

\newcommand{\chapter}[2]{%
\setcounter{section}{#1-1}%
\section{#2}%
}

\newcommand{\subchapter}[2]{%
\setcounter{subsection}{#1-1}%
\subsection{#2}%
}

\newcommand{\problem}[1]{%
\setcounter{subsubsection}{#1-1}%
\subsubsection{\hfill}%
}

\newcommand{\solution}{%
\subsubsection*{Solution}%
}

% Misc math operators

\DeclarePairedDelimiter{\ceil}{\lceil}{\rceil}
\DeclarePairedDelimiter{\floor}{\lfloor}{\rfloor}

% Operators for predicate logic

\DeclareMathOperator{\T}{\text{\textbf T}}
\DeclareMathOperator{\F}{\text{\textbf F}}
\DeclareMathOperator{\lthen}{\to}
\DeclareMathOperator{\limplies}{\to}
\DeclareMathOperator{\lwhen}{\gets}
\DeclareMathOperator{\lif}{\gets}
\DeclareMathOperator{\liff}{\leftrightarrow}
\DeclareMathOperator{\lxor}{\oplus}

% Sets

\DeclarePairedDelimiter{\set}
	{\lbrace}
	{\rbrace}
\DeclareMathOperator{\ZZ}{\mathbb{Z}}
\DeclareMathOperator{\SetOfIntegers}{\ZZ}
\DeclareMathOperator{\ZZPos}{\mathbb{Z}^+}
\DeclareMathOperator{\SetOfPositiveIntegers}{\ZZPos}
\DeclareMathOperator{\NN}{\mathbb{N}}
\DeclareMathOperator{\SetOfNaturalNumbers}{\NN}
\DeclareMathOperator{\RR}{\mathbb{R}}
\DeclareMathOperator{\SetOfRealNumbers}{\RR}
\DeclareMathOperator{\RRPos}{\mathbb{R}^+}
\DeclareMathOperator{\SetOfPositiveRealNumbers}{\RRPos}
\DeclareMathOperator{\QQ}{\mathbb{Q}}
\DeclareMathOperator{\SetOfRationalNumbers}{\QQ}
\DeclareMathOperator{\CC}{\mathbb{C}}
\DeclareMathOperator{\SetOfComplexNumbers}{\CC}

\begin{document}

\title{Exercise Class Solutions 3}
\date{}
\author{}
\maketitle

\chapter{13}{Modeling Computation}
	\subchapter{2}{Finite-State Machines with Output}
		\subfile{problems/13.2.9.tex}
		\pagebreak
	\subchapter{3}{Finite-State Machines with no Output}
		\subfile{problems/13.3.1abcd.tex}
		\pagebreak
		\subfile{problems/13.3.5abcd.tex}
		\pagebreak
		\subfile{problems/13.3.9cde.tex}
		\pagebreak
		\subfile{problems/13.3.17.tex}
		\pagebreak
		\subfile{problems/13.3.19.tex}
		\pagebreak
		\subfile{problems/13.3.27.tex}
		\pagebreak
		\subfile{problems/13.3.29.tex}
		\pagebreak
		\subfile{problems/13.3.31.tex}
		\pagebreak
		\subfile{problems/13.3.35.tex}
		\pagebreak
		\subfile{problems/13.3.25.tex}
		\pagebreak
		\subfile{problems/13.3.39.tex}
		\pagebreak
		\subfile{problems/13.3.41.tex}
		\pagebreak
		\subfile{problems/13.3.43.tex}
		\pagebreak
		\subfile{problems/13.3.45.tex}
		\pagebreak

\end{document}

%
% Ragnar Ardal
%
% Logic in Computer Science
%
% Reykjavik University 2016
%

\documentclass[11pt,a4paper]{article}

%\usepackage{multicol}

\usepackage{amsmath}
\usepackage{amssymb}
%\usepackage{amsthm}

\usepackage{mathtools}

\usepackage{enumerate}

% Organization of LaTeX file

\newcommand{\chapter}[2]{%
\setcounter{section}{#1}%
\addtocounter{section}{-1}%
\section{#2}%
}
\newcommand{\subchapter}[2]{%
\setcounter{subsection}{#1}%
\addtocounter{subsection}{-1}%
\subsection{#2}%
}
\newcommand{\problem}[1]{%
\setcounter{subsubsection}{#1}%
\addtocounter{subsubsection}{-1}%
\subsubsection{\hfill}%
}
\newenvironment{subproblem}
	{\begin{enumerate}[a)]}
	{\end{enumerate}}
\newcommand{\skipitem}{\addtocounter{enumi}{1}}
\newcommand{\solution}{%
\subsubsection*{Solution}%
}

% Misc math operators

\DeclarePairedDelimiter{\ceil}{\lceil}{\rceil}
\DeclarePairedDelimiter{\floor}{\lfloor}{\rfloor}

% Operators for predicate logic

\DeclareMathOperator{\T}{\text{\textbf T}}
\DeclareMathOperator{\F}{\text{\textbf F}}
\DeclareMathOperator{\lthen}{\to}
\DeclareMathOperator{\limplies}{\to}
\DeclareMathOperator{\lwhen}{\gets}
\DeclareMathOperator{\lif}{\gets}
\DeclareMathOperator{\liff}{\leftrightarrow}
\DeclareMathOperator{\lxor}{\oplus}

% Sets

\DeclarePairedDelimiter{\set}
	{\lbrace}
	{\rbrace}
\DeclareMathOperator{\ZZ}{\mathbb{Z}}
\DeclareMathOperator{\SetOfIntegers}{\ZZ}
\DeclareMathOperator{\ZZPos}{\mathbb{Z}^+}
\DeclareMathOperator{\SetOfPositiveIntegers}{\ZZPos}
\DeclareMathOperator{\NN}{\mathbb{N}}
\DeclareMathOperator{\SetOfNaturalNumbers}{\NN}
\DeclareMathOperator{\RR}{\mathbb{R}}
\DeclareMathOperator{\SetOfRealNumbers}{\RR}
\DeclareMathOperator{\RRPos}{\mathbb{R}^+}
\DeclareMathOperator{\SetOfPositiveRealNumbers}{\RRPos}
\DeclareMathOperator{\QQ}{\mathbb{Q}}
\DeclareMathOperator{\SetOfRationalNumbers}{\QQ}
\DeclareMathOperator{\CC}{\mathbb{C}}
\DeclareMathOperator{\SetOfComplexNumbers}{\CC}

\begin{document}

\title{TA Solutions 5}
\date{}
\author{}
\maketitle

\chapter{2}{Basic Structures: Sets, Functions, Sequences, Sums, and Matrices}
	\subchapter{4}{Sequences and Summations}
		
%
% Worked by AUTHOR
%
% Page numbers:
%  -    p - 7th Custom Edition
%  -    p - Icelandic Edition
%  -    p - 7th Edition (US)
%  -    p - 6th Edition
%  -    p - 7th Global Edition
%

\problem{}
	Problem description
	\begin{subproblem}
		\skipitem
		\item 
	\end{subproblem}

\solution


		\pagebreak
		
%
% Worked by AUTHOR
%
% Page numbers:
%  -    p - 7th Custom Edition
%  -  210 - Icelandic Edition
%  -    p - 7th Edition (US)
%  -    p - 6th Edition
%  -    p - 7th Global Edition
%

\problem{19}
	Suppose that the number of bacteria in a colony triples every hour.
	\begin{subproblem}
		\item Set up a recurrence relation for the number of bacteria after $n$ hours have elapsed.
		\item If 100 bacteria are used to begin a new colony, how many bacteria will be in the colony in 10 hours?
	\end{subproblem}

\solution
	\begin{subproblem}
		\item 
		\item 
	\end{subproblem}

		\pagebreak
		
%
% Worked by AUTHOR
%
% Page numbers:
%  -    p - 7th Custom Edition
%  -  211 - Icelandic Edition
%  -    p - 7th Edition (US)
%  -    p - 6th Edition
%  -    p - 7th Global Edition
%

\problem{29}
	
	\begin{subproblem}
		\item $\sum\limits_{k=1}^5 (k + 1)$
		\skipitem
		\item $\sum\limits_{i=1}^{10} 3$
	\end{subproblem}

\solution
	\begin{subproblem}
		\item 
		\skipitem
		\item 
	\end{subproblem}

		\pagebreak
		
%
% Worked by AUTHOR
%
% Page numbers:
%  -    p - 7th Custom Edition
%  -    p - Icelandic Edition
%  -    p - 7th Edition (US)
%  -    p - 6th Edition
%  -    p - 7th Global Edition
%

\problem{}
	Problem description
	\begin{subproblem}
		\skipitem
		\item 
	\end{subproblem}

\solution


		\pagebreak
		
%
% Worked by AUTHOR
%
% Page numbers:
%  -    p - 7th Custom Edition
%  -    p - Icelandic Edition
%  -    p - 7th Edition (US)
%  -    p - 6th Edition
%  -    p - 7th Global Edition
%

\problem{}
	Problem description
	\begin{subproblem}
		\skipitem
		\item 
	\end{subproblem}

\solution


		\pagebreak
		
%
% Worked by AUTHOR
%
% Page numbers:
%  -    p - 7th Custom Edition
%  -  211 - Icelandic Edition
%  -    p - 7th Edition (US)
%  -    p - 6th Edition
%  -    p - 7th Global Edition
%

\problem{39}
	Find $\sum_{k=100}^{200} k$. (Use Table 2.)

\solution
	

		\pagebreak
	\subchapter{6}{Matrices}
		
%
% Worked by AUTHOR
%
% Page numbers:
%  -    p - 7th Custom Edition
%  -    p - Icelandic Edition
%  -    p - 7th Edition (US)
%  -    p - 6th Edition
%  -    p - 7th Global Edition
%

\problem{}
	Problem description
	\begin{subproblem}
		\skipitem
		\item 
	\end{subproblem}

\solution


		\pagebreak
		
%
% Worked by AUTHOR
%
% Page numbers:
%  -    p - 7th Custom Edition
%  -  229 - Icelandic Edition
%  -    p - 7th Edition (US)
%  -    p - 6th Edition
%  -    p - 7th Global Edition
%

\problem{3}
	Find $\mathbf{AB}$ if
	\begin{subproblem}
		\item $\mathbf{A} = \begin{bmatrix*}[r] 2 & 1 \\ 3 & 2 \end{bmatrix*}$,
			  $\mathbf{B} = \begin{bmatrix*}[r] 0 & 4 \\ 1 & 3 \end{bmatrix*}$.
		\item $\mathbf{A} = \begin{bmatrix*}[r] 1 &-1 \\ 0 & 1 \\ 2 & 3 \end{bmatrix*}$,
			  $\mathbf{B} = \begin{bmatrix*}[r] 3 &-2 &-1 \\ 1 & 0 & 2 \end{bmatrix*}$,
	\end{subproblem}

\solution
	\begin{subproblem}
		\item 
		\item 
	\end{subproblem}

		\pagebreak
		
%
% Worked by Ragnar Ardal
%
% Page numbers:
%  -    p - 7th Custom Edition
%  -  230 - Icelandic Edition
%  -    p - 7th Edition (US)
%  -    p - 6th Edition
%  -    p - 7th Global Edition
%

\problem{11}
	What do we know about the sizes of the matrices $\mathbf{A}$ and $\mathbf{B}$ if both of the products $\mathbf{AB}$ and $\mathbf{BA}$ are defined?

\solution
	That both are squares of equal size.

		\pagebreak
		
%
% Worked by AUTHOR
%
% Page numbers:
%  -    p - 7th Custom Edition
%  -  230 - Icelandic Edition
%  -    p - 7th Edition (US)
%  -    p - 6th Edition
%  -    p - 7th Global Edition
%

\problem{15}
	Let
	$$\mathbf{A} = \begin{bmatrix} 1 & 1 \\ 1 & 1 \end{bmatrix}$$.
	Find a formula for $\mathbf{A}^n$,
	whenever $n$ is a positive integer.

\solution
	


\end{document}
